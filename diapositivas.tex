---
marp: true
paginate: true
size: 4:3
---

# Desarrollo web independiente
###### Alberto Benavides


San Nicolás, N. L. 
20 de noviembre de 2019    

---

## Temario

- Conceptos fundamentales
- Consejos basados en mi experiencia
- Recomendaciones técnicas
- Mini tutorial

---

# Conceptos fundamentales

---

## Página web VS Aplicación web

Página web | Aplicación web
---|---
Publicitaria | Administrativa
Sin acceso de usuarios | Con acceso de usuarios
Contenido estático | Contenido dinámico
Visual | Interactiva
Bajo coste | Elevado coste
Poco tiempo | Mucho tiempo
Mínima responsabilidad | Gran responsabilidad

---

## Servidor

![h:600](https://www.progress.com/documentation/sitefinity-cms/sf-images/default-source/default-album/architecture-diagram-8.jpg?sfvrsn=0)

---

## Peticiones

- Get
	- Petición por barra de búsqueda
	`http://ruta.com?llave=valor`
	- Compartir información
	- Identificar recursos
	- Tamaño limitado (no archivos multimedia)
- Post
	- Petición en el cuerpo del mensaje
	- Oculta
	- Información confidencial
	- Tamaño "ilimitado" (servidor)

---

# Consejos basados en mi experiencia

---

## Habilidades ideales

- Programador: Conoce distintos lenguajes (mín. 3)
- Artista visual: Crea contenidos estéticos
- Mercadólogo: Crea contenidos intuitivos
- Matemático: Resuelve problemas con  algoritmos
- Vendedor: Negocia precios con los clientes

---

## Ventajas

* Libertad de horarios y "oficinas"
* Distintos proyectos = Mucho aprendizaje
* Aprendes a negociar = Valorar tu trabajo
* Dinero negociado = Dinero ganado

---

## Desventajas

* Hay que convencer al cliente
* ¡Hay que aprender a cobrar! `https://www.cuantocuestamiweb.com/`
* Proyectos grandes: Necesidad de equipo multidisciplinario

---

## Recomendaciones

- Ser sincero: Capacidades, inconvenientes, limitaciones
- No aceptar trabajos mal pagados
- Establecer un documento con entregables 

---

# Recomendaciones técnicas

---

## Software

- Visual Studio Code: Editor de textos pensado para programadores
- XAMPP: Plataforma independiente para servidores locales
- Sourcetree: Manejador visual de control de versiones
- Consola o terminal: `cd`

---

## Tecnologías indispensables

- HTML
	- *Hiper Text Markup Language*
	- Lenguaje de formateo de texto
	- No es un lenguaje de programación
- CSS
	- *Cascading Style Sheet*
	- Hojas de estilo de cascadeo
	- Estilo de páginas web
- JS: Lenguaje de programación integrado en navegadores

---

## Paquete básico

- PHP
	- *Programming Hypertext Preprocessor*
	- Lenguaje de programación para HTML incrustado
- jQuery
	- Librería de JS
	- Facilita la manipulación de los elementos HTML
	- Añade atajos de funciones de JS
- MySQL:
	- Base de datos relacional
	- Funciona con tablas
- Bootstrap: Plantilla de diseño utilizada por Twitter

---

# Mini tutorial de HTML a Laravel

---

## Estructura mínima HTML

```html
<html>
  <head>
    <title>Ejemplo HTML</title>
    <style>
      #rojo{
        color: red;
      }
   </style>
  </head>
  <body>
    <h1>Características</h1>
	<ul>
	  <li>Usa etiquetas</li>
	  <li id="rojo">Estilos por selectores</li>
	</ul>
	<script>
      alert("Mira esta alerta.");
	</script>
  </body>
</html>
```

---

## Recomendación [Laravel](https://laravel.com/)

* *Framework* PHP = Marco de trabajo en PHP
* Conjunto de programas, métodos, instrucciones... que realizan una tarea específica
* Tarea específica: Crear una aplicación web

![](https://nordicapis.com/wp-content/uploads/Laravel-logo.png)

---

## Requisitos
* [Git](https://git-scm.com/): Software de control de versiones
* [Composer](https://getcomposer.org/): Herramienta para manipulación de dependencias de PHP

![](https://git-scm.com/images/logos/downloads/Git-Icon-1788C.png) ![](https://getcomposer.org/img/logo-composer-transparent4.png)

---

## Instalación


```
composer create-project --prefer-dist laravel/laravel nombreApp
```

---

## Servidor local

```
php artisan serve

rem Entrar a http://localhost:8000
```

---

## Archivo `.env`

* (Revisar) Renombrar archivo `.env.example` a `.env`
* Contiene la información del entorno de la aplicación
* Obtener variables de entorno

```php
// El segundo parámetro de env() es el valor por defecto
'debug' => env('APP_DEBUG', false), // en config/app.php

$environment = App::environment();

```

---

## Llave de la aplicación

Permite asegurar sesiones y datos encriptados

1. Generar llave de la aplicación

```
php artisan key:generate
```

2. Agregar la llave al archivo `.env`
```
APP_KEY=base64:...
```

---

## Carpetas principales

* `app`: Modelos de tablas (Usuario)
* `config`: Todas las configuraciones de la aplicación
* `database`: 
	* `migrations`: Especificación de tablas (usuarios)
	* `seeds`: Registros predeterminados de tablas
* `public`:
	* `css`, `js`: Archivos de estilo y scripts
    * `index.php`: Llegada de peticiones de usuarios
    
---

* `resources`:
	* `lang`: Contiene los idiomas instalados
	* `views`: Contiene las vistas (páginas) de la aplicación
* `routes/web.php`: Contiene las redirecciones de la aplicación
* `storage`: Archivos generados por cada sesión
	* `storage/app/public`: Usada generalmente para almacenar archivos generados por el usuario

---

## Creación de base de datos con MySQL

1. Abrir XAMPP
2. Ejecutar MySQL
3. Entrar al shell de XAMPP
4. Entrar a la base de datos:
`mysql -u root -p`
5. Crear base de datos
`create database ejemplo;`
6. Actualizar `.env`

---

## Configuración de autentificación

```php
// app/Providers/AppServiceProvider.php

use Illuminate\Support\Facades\Schema;

public function boot()
{
    Schema::defaultStringLength(191);
}
```

```
rem cd carpeta/del/proyecto

rem Prepara vistas, modelos y migrations para auth
composer require laravel/ui --dev
npm install && npm run dev
php artisan ui vue --auth

rem Corre las migraciones
php artisan migrate
```

---

## Subir repositorio a GitHub

1. Crear cuenta en [GitHub](http://www.github.com)
2. Clic en + (arriba a la derecha) -> New repository
4. Agregar un nombre
5. Create repository
6. `git init`
9. `git remote add origin url`

Cada cambio significativo:

7. `git add *`
8. `git commit -m "Mensaje"`
10. `git push origin master`

---

## Referencias

- https://www.progress.com/documentation/sitefinity-cms/reference-architecture-diagrams
- https://developer.mozilla.org/es/docs/Web/HTML
- https://developer.mozilla.org/es/docs/Web/CSS
- https://developer.mozilla.org/es/docs/Learn/JavaScript/First_steps/Qu%C3%A9_es_JavaScript
- https://jquery.com/
- https://laravel.com/
- https://vuejs.org/